\documentclass[12pt]{article}

\usepackage[utf8]{inputenc}      
\usepackage[T1]{fontenc}          
\usepackage[a4paper, margin=2.5cm]{geometry}  % page size & margins
\usepackage{setspace}             % spacing 
\usepackage{graphicx}             % images
\usepackage{hyperref}             
\usepackage[backend=biber, style=apa]{biblatex} % citing
\usepackage{fontspec} % font

% Settings
\doublespacing
\addbibresource{references.bib}  

\setmainfont{Times New Roman}
\setmonofont{Lucida Console}[Scale=MatchLowercase]

% ===== Begin Document =====
\begin{document}
	
	\begin{titlepage}
		\centering
		
		\textbf{Global Politics SL \textemdash Engagement Activity}
		
		\vspace*{4cm}
		
		\textbf{Title:}\\
		Negotiating Security and Rights: Public Surveillance in Darmstadt
		
		\vspace{1cm}
		
		\textbf{RQ:}\\
		To what extent does the use of public surveillance in Darmstadt challenge the balance between collective security and individual rights?
		
		\vspace{4cm}
		
		Word count:\\
		XXXX
		
		\vfill
	\end{titlepage}
	
	
	In an increasingly digital and urbanized world, the use of public surveillance technologies in urban spaces has become widely ubiquitous. Justified by the promise of a collective security, the safeguarding of multiple people against threats ranging from domestic crime to international attacks \parencites{noauthor_security_2025}{noauthor_collective_2025}, these measures are particularly employed in areas marked by high foot traffic or frequent public unrest. However, the proliferation of these surveillance systems provokes concerns about the integrity of individual rights \textemdash\ in particular the right to privacy, to assembly and associate with others and the freedom of expression \parencite{nandy2023}. Striking an equilibrium between these seemingly incongruous concerns remains a central challenge in modern democracies, notably because freedom of expression constitutes one of their fundamental cornerstones \parencite{noauthor_special_nodate}. This issue is markedly salient in Germany, where the historical legacy of state surveillance \textemdash\ both under the Nazi regime \parencite{mdrde_uberwachung_nodate} and later the GDR \parencite{lichter_loeffler_siegloch2016} \textemdash\ continues to shape public conscience.
	
	The city of Darmstadt poses as an ideal case study to explore this blurred divide further. Labeled as a ``Smart City'' \parencite{noauthor_smart_nodate}, Darmstadt is home to the Luisenplatz, 
	
	
	
	
	\clearpage
	
	\printbibliography
	
\end{document}
