\documentclass[12pt]{article}

\usepackage[utf8]{inputenc}      
\usepackage[T1]{fontenc}          
\usepackage[a4paper, margin=2.5cm]{geometry}  % page size & margins
\usepackage{setspace}             % spacing 
\usepackage{graphicx}             % images
\usepackage{hyperref}             
\usepackage[backend=biber, style=apa]{biblatex} % citing
\usepackage{fontspec} % font

% Settings
\doublespacing
\addbibresource{references.bib}  

\setmainfont{Times New Roman}
\setmonofont{Lucida Console}[Scale=MatchLowercase]

% ===== Begin Document =====
\begin{document}
	
	\begin{titlepage}
		\centering
		
		\textbf{Global Politics SL \textemdash\ Engagement Activity}
		
		\vspace*{4cm}
		
		\textbf{Title:}\\
		Negotiating Security and Rights: Public Surveillance in Darmstadt
		
		\vspace{1cm}
		
		\textbf{RQ:}\\
		To what extent does the use of public surveillance in Darmstadt challenge the balance between collective security and individual rights?
		
		\vspace{4cm}
		
		Word count:\\
		XXXX
		
		\vfill
	\end{titlepage}
	
	
	In an increasingly digital and urbanized world, the use of public surveillance technologies in urban spaces has become widely ubiquitous. Justified by the promise of a collective security \textemdash\ the safeguarding of multiple people against threats ranging from domestic crime to international attacks \parencites{noauthor_security_2025}{noauthor_collective_2025} \textemdash\ these measures are particularly employed in areas marked by high foot traffic and frequent public unrest. However, the proliferation of these surveillance systems provokes concerns about the integrity of individual rights \textemdash\ in particular the right to privacy, to assembly and associate with others and the freedom of expression \parencite{nandy2023}. Striking an equilibrium between these seemingly incongruous concerns remains a central challenge in modern democracies, where freedom of expression \textemdash\ one of demoncracy's fundamental cornerstones \parencite{noauthor_special_nodate} \textemdash\ stands vulnerable. 
		
	Germany, embossed by its historical legacy of state surveillance \textemdash\ under the Nazi regime \parencite{mdrde_uberwachung_nodate} and the GDR \parencite{lichter_loeffler_siegloch2016} \textemdash\ has developed a multifaceted and intricate legal framework governing this issue. Within this context, Darmstadt emerges as an ideal case study.
	
	Home to the Luisenplatz \textemdash\ the central square and primary transfer hub for the area's public transportation network \textemdash\ Darmstadt embodies the urban environment where surveillance is most prominent. I have personally traversed this square countless times \textemdash\ on my way to meet up with friends or to go shopping in the vast variety of stores offered \textemdash\ and became increasingly conscious of the surveillance cameras. Realizing that any person residing on the square was being recorded without explicit consent led me to question the legitimacy of these systems \textemdash\ culminating into the research question of this paper: \textit{To what extent does the use of public surveillance in Darmstadt challenge the balance between collective security and individual rights?}.
	
	This investigation is deeply anchored in the core concepts of the Global Politics course, particularly \textit{Power}. The implementation of surveillance exemplifies state power in both its hard and soft forms \textemdash\ asymmetrical collection of information in order to pursue objectives such as maintaining public order, and through the potential ``chilling effect'' induced by state observation \parencite{murray2024}. Furthermore, it is also related to the thematic studies \textit{Peace and Conflict} and \textit{Rights and Justice}. Proponents of surveillance argue it is essential for ensuring peace and shielding the population from violence. Nevertheless, the utilization surveillance systems must be weighed against human rights considerations.
	
	By examining both the German legislation regarding this dilemma and exploring how surveillance is handled, implemented and perceived by the public in Darmstadt, this analysis operates both on the national and local level. Nonetheless, the local level remains the focal point, with primary information gained from two interviews: an interview with Prof. Dr. Dr. Christian Reuter, Chair of PEASEC \textemdash\ an organization which merges computer science with peace and security studies \parencite{noauthor_peasec_2025} \textemdash\ and professor at the Technische Universität Darmstadt, and Paul Georg Wandrey, Chair of the Darmstadt City Council.
	
	The interview with Reuter was chosen due to his vast expertise concerning the intersection of technology, peace and security, and democratic values specific to Darmstadt, allowing for a profound insight on the justifications and ramifications of surveillance. In contrast to Wandrey, Reuter is independent of political interests, therefore providing a critical yet neutral perspective on the matter. While highlighting the importance of proportionality, Reuter argued that a perfect balance is challenging to define due to the difficulty to measure the importance of values. However, a lack of specific details prevented him from providing concrete statements.
	
	The interview with Wandrey was selected due to his position at the forefront of the local decision-making process, providing crucial insight into the political rationale behind the use of surveillance. As part of the City Council, Wandrey has access to precise information regarding the implementation of and feedback concerning surveillance in Darmstadt \textemdash\ offsetting the shortcomings regarding concrete information of the interview with Reuter. In agreement with Reuter, Wandrey emphasized the concept of proportionality and supported regulated artificial intelligence driven surveillance systems. Moreover, he highlighted the paradox of usage of social media and explained measures facilitating an elevated level of privacy.
	
	Both interviews were video conferences conducted via Microsoft Teams \textemdash\ in exchange for in-person interviews \textemdash\ acknowledging its flexibility and greater time efficiency, improving the likelihood of securing an interview. A list of questions formulated before the interviews acted as guidance. The interviews were recorded \textemdash\ with consent \textemdash\ and transcribed for later reference.
	
	At face value, the mere employment of surveillance undermines 
	
	
	
	
	
	
	
		
	
	
	
	
	
	
	
	
	
	\clearpage
	
	\printbibliography
	
\end{document}
