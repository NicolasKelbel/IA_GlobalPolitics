\documentclass[12pt]{article}

\usepackage[utf8]{inputenc}      
\usepackage[T1]{fontenc}          
\usepackage[a4paper, margin=2.5cm]{geometry}  % page size & margins
\usepackage{setspace}             % spacing 
\usepackage{graphicx}             % images
\usepackage{hyperref}             
\usepackage[backend=biber, style=apa]{biblatex} % citing
\usepackage{fontspec} % font

% Settings
\doublespacing
\addbibresource{references.bib}  

\setmainfont{Times New Roman}
\setmonofont{Lucida Console}[Scale=MatchLowercase]

% ===== Begin Document =====
\begin{document}
	
	\begin{titlepage}
		\centering
		
		\textbf{Global Politics SL \textemdash\ Engagement Activity}
		
		\vspace*{4cm}
		
		\textbf{Title:}\\
		Negotiating Security and Rights: Public Surveillance in Darmstadt
		
		\vspace{1cm}
		
		\textbf{RQ:}\\
		To what extent does the use of public surveillance in Darmstadt challenge the balance between collective security and individual rights?
		
		\vspace{4cm}
		
		Word count:\\
		XXXX
		
		\vfill
	\end{titlepage}
	
	
	In an increasingly digital and urbanized world, the use of public surveillance technologies in urban spaces has become widely ubiquitous. Justified by the promise of a collective security \textemdash\ the safeguarding of multiple people against threats ranging from domestic crime to international attacks \parencites{noauthor_security_2025}{noauthor_collective_2025} \textemdash\ these measures are particularly employed in areas marked by high foot traffic and frequent public unrest. However, the proliferation of these surveillance systems provokes concerns about the integrity of individual rights \textemdash\ in particular the right to privacy, to assembly and associate with others and the freedom of expression \parencite{nandy2023}. Striking an equilibrium between these seemingly incongruous concerns remains a central challenge in modern democracies. 
	
	Germany, embossed by its historical legacy of state surveillance \textemdash\ under the Nazi regime \parencite{mdrde_uberwachung_nodate} and the GDR \parencite{lichter_loeffler_siegloch2016} \textemdash\ has developed a multifaceted and intricate legal framework governing this issue. Within this context, Darmstadt emerges as an ideal case study.
	
	Home to the Luisenplatz \textemdash\ the civic square and the central transfer hub for the area's public transportation network \textemdash\ Darmstadt epitomizes the urban environment where surveillance is most prominent. I have personally traversed this square countless times \textemdash\ on my way to meet up with friends or to go shopping in the vast variety of stores offered \textemdash\ and became increasingly conscious of the surveillance cameras. Realizing that any person residing on the square was being recorded without explicit consent led me to question the oversight and ethics of these systems \textemdash\ culminating into the research question of this paper: \textit{To what extent does the use of public surveillance in Darmstadt challenge the balance between collective security and individual rights?}.
	
	This investigation is deeply anchored in the core concepts of the Global Politics course, particularly power, security and human rights. The implementation of surveillance exemplifies state power by 
	
	\textemdash\ the pursuit of maintaining public order and shielding its populace.
	
	To explore the matter I...
	
	
	
	
	
	
	
	
	\clearpage
	
	\printbibliography
	
\end{document}
