\documentclass[12pt]{article}

\usepackage[utf8]{inputenc}      
\usepackage[T1]{fontenc}          
\usepackage[a4paper, margin=2.5cm]{geometry}  % page size & margins
\usepackage{setspace}             % spacing 
\usepackage{graphicx}             % images
\usepackage{hyperref}             
\usepackage[backend=biber, style=apa]{biblatex} % citing
\usepackage{fontspec} % font
\usepackage[toc]{appendix}
\usepackage{float} % for figure positioning
\usepackage[font=small,labelfont=bf]{caption}
\usepackage[most]{tcolorbox}
\usepackage[noabbrev,nameinlink]{cleveref}
\usepackage{enumitem}


% Settings
\doublespacing
\addbibresource{references.bib}  

\setmainfont{Times New Roman}
\setmonofont{Lucida Console}[Scale=MatchLowercase]

\newcommand{\quotecard}[2]{%
	\begin{center}
		\begin{minipage}{0.93\linewidth}
			\hrule\vspace{6pt}
			#2
			\vspace{6pt}\hrule
		\end{minipage}
	\end{center}
}

% ===== Begin Document =====
\begin{document}
	
	\begin{titlepage}
		\centering
		
		\textbf{Global Politics SL \textemdash\ Engagement Activity}
		
		\vspace*{4cm}
		
		\textbf{Title:}\\
		Negotiating Security and Rights: Public Surveillance in Darmstadt
		
		\vspace{1cm}
		
		\textbf{RQ:}\\
		To what extent does the use of public surveillance in Darmstadt challenge the balance between collective security and individual rights?
		
		\vspace{4cm}
		
		Word count:\\
		1998
		
		\vfill
	\end{titlepage}
	
	
	In an increasingly digital and urbanized world, the use of public surveillance technologies in urban spaces has become ubiquitous. Justified by the promise of collective security \textemdash\ the safeguarding of multiple people against threats ranging from domestic crime to international attacks \parencites{noauthor_security_2025}{noauthor_collective_2025} \textemdash\ these measures are particularly employed in areas marked by high foot traffic. However, the proliferation of these surveillance systems provokes concerns about the integrity of individual rights; in particular the right to privacy, to assembly and associate with others, the freedom of expression and principles of equality and non-discrimination \parencite{nandy2023}. Striking an equilibrium between these seemingly incongruous concerns remains a central challenge in modern democracies, where one of their fundamental cornerstones \textemdash\ freedom of expression \textemdash\ stands vulnerable \parencite{noauthor_special_nodate}. 
		
	Germany, embossed by its historical legacy of state surveillance under the Nazi regime \parencite{mdrde_uberwachung_nodate} and the German Democratic Republic \parencite{lichter_loeffler_siegloch2016}, has developed a multifaceted and intricate legal framework governing this issue. Within this context, Darmstadt emerges as an ideal case study.
	
	Home to the Luisenplatz \textemdash\ the central square and primary transfer hub for the area's public transportation network \textemdash\ Darmstadt embodies the urban environment where surveillance is most prominent, with the square alone accounting for close to 175 criminal offenses across 2023 and the first half of 2024 \parencite{schonbein_weapon_2025}. Having personally traversed this square countless times, on my way to meet with friends, I became increasingly conscious of the surveillance cameras (see \Cref{sec:luisenplatz-map}). Realizing that anyone present on the square was being recorded without explicit consent led me to question the legitimacy of these systems, ultimately culminating into the research question of this paper: \textit{To what extent does the use of public surveillance in Darmstadt challenge the balance between collective security and individual rights?}.
	
	This investigation is deeply anchored in the core concepts of the Global Politics course, particularly \textit{Power} and \textit{Legitimacy}. The implementation of surveillance exemplifies power of political actors in both its hard and soft forms \parencite{courseCompanion2024}\ \textemdash\ the collection of information in order to pursue objectives such as maintaining public order, and through the potential ``chilling effect'' induced by state observation \parencite{murray2024}. One is confronted with the legitimacy of the authority of these political actors, examining whether their actions retain the democratic values instilled within Germany's constitution. Furthermore, it is also related to the thematic studies \textit{Peace and Conflict} and \textit{Rights and Justice}. Proponents of surveillance argue it is essential for ensuring peace and shielding the population from violence. Nevertheless, the utilization surveillance systems must be weighed against human rights considerations.
	
	By examining both the German legislation regarding this dilemma and exploring how surveillance is handled, implemented and perceived by the public in Darmstadt, this analysis operates both on the national and local level. Nonetheless, the local level remains the focal point, with primary information gained from two interviews: an interview with Prof. Dr. Dr. Christian Reuter, Chair of PEASEC \textemdash\ an organization which merges computer science with peace and security studies \parencite{noauthor_peasec_2025} \textemdash\ and professor at the Technische Universität Darmstadt, and Paul Georg Wandrey, Chair of the Darmstadt City Council.
	
	The interview with Reuter was chosen due to his vast expertise concerning the intersection of technology, peace and security, and democratic values specific to Darmstadt, allowing for a profound insight on the justifications and ramifications of surveillance. In contrast to Wandrey, Reuter is independent of political interests, therefore providing a critical yet neutral perspective on the matter. While highlighting the importance of proportionality, Reuter argued that a perfect balance is challenging to define due to the difficulty to measure the importance of values. However, a lack of specific details prevented him from providing concrete statements.
	
	The interview with Wandrey was selected due to his position at the forefront of the local decision-making process, providing crucial insight into the political rationale behind the use of surveillance. As part of the City Council, Wandrey has access to precise information regarding the implementation of and feedback concerning surveillance in Darmstadt \textemdash\ offsetting the shortcomings regarding concrete information of the interview with Reuter. In agreement with Reuter, Wandrey emphasized the concept of proportionality and supported regulated artificial intelligence driven surveillance systems. Moreover, he highlighted the paradox of usage of social media and explained measures facilitating an elevated level of privacy.
	
	Both interviews were video conferences conducted via Microsoft Teams for reasons of flexibility and time efficiency, improving the likelihood of securing an interview. A list of questions formulated before the interviews acted as guidance. The interviews were recorded with consent and transcribed for later reference (see \Cref{app:interviews}). 
	
	At face value, the mere employment of surveillance undermines various inherent human rights. The collection of private data, interaction logs and activity patterns through surveillance partially or even fully undermines an individual's right to privacy, challenging Articles 1(1) and 10(1) of the \textit{Grundgesetz} (German Basic Law) \parencites{nandy2023, wetzling2023, grundgesetz2025}. Furthermore, the omnipresent nature of surveillance may lead to ``self-censor[ship]'', suppressed public dialog and the restraining of free and open discourse \textemdash\ ``chilling effect'' \textemdash\ and contests Article 5(1) by compromising the right of freedom of expression \parencite{murray2024}. A reluctance to participate in non-violent demonstrations or civil society organizations may develop out of the concern of becoming a potential target, curtailing the right to assemble and associate with others under Articles 8(1) and 9(1). Lastly, surveillance systems may exhibit structural bias, possibly diminishing principles of equality and non-discrimination given by Article 3(3). When viewed through the lens of liberalism, especially the interpretation proposed by Judith Shklar \textemdash\ liberalism's main objective is to ensure a political landscape in which individuals are able to exercise their personal freedom \textemdash\ such surveillance systems raise existential concerns about the erosion of the very freedoms the state is meant to protect \parencite[684]{bell2014}.
	
	To expand on the notion of the ``chilling effect'', Reuter points out a thought-provoking implication of surveillance: fabrication of distorted realities. Partial quotes or brief excerpts of surveillance footage could be employed in order to create false narratives revolving around an individual or group of people. Beyond the use of false narratives to diminish public perception of a given target, they may serve as basis for unjust incrimination. This potential to manipulate events intensifies the ``chilling effect'', as the concern about being subject to targeted scrutiny rises significantly. 
	
	However, as Reuter also notes, it is quintessential to also consider the potentially advantageous impact of surveillance on certain individual liberties. Article 2(2) of the Grundgesetz guarantees the right to \textit{Körperliche Unversehrheit} (Physical Integrity), protecting individuals from bodily harm, torture and other infringements on physical integrity \parencite{bildung_korperliche_nodate}. Given that surveillance primarily functions to both deter and investigate criminal activity, as underscored by Wandrey, it serves to strengthen the collective security of the populace, thereby affirming the protection of the right to \textit{Körperliche Unversehrheit}. Placing heightened focus on security is in line with defensive realism \parencite{courseCompanion2024}, a political theory opposing liberalism \parencite{jumarang2011}.  
	
	Furthermore, as Wandrey stresses, it is highly necessary to examine the regulations under which surveillance in Darmstadt is conducted. With a daily foot traffic of well above 100,000 individuals, a figure provided by Wandrey, Luisenplatz can be considered a major public hotspot, thus providing justification given the amplified probability of criminal activities. The requirement to justify surveillance measures every two years ensures that they are reflective of the present circumstances and evolving societal needs. In an attempt to combat the ``chilling effect'', demonstrations or assemblies can be registered \textemdash\ without the need for approval \textemdash\ which automatically triggers the deactivation of surveillance through a physical cover during the registered time period. Hospitality zones are not surveilled during operation hours, providing their customers with elevated privacy. Lastly, as per Article 4(5) of the Bundesdatenschutzgesetz (German Federal Data Protection Act), there are defined periods for which the surveillance footage may be stored and access to it is heavily restricted, further promoting the privacy of the individual \parencite{gesetzeimInternet_BDSG4}. 
	
	A particularly compelling and often overlooked argument mentioned by Wandrey concerns the paradoxical relationship between public surveillance and the widespread usage of social media. In Germany, the majority of the population use social media platforms \parencite{koptyug2025, worldometer_germany2025}, and by doing so, voluntarily disclosing vast amounts of personal data to multinational technology giants such as Meta. Unlike governmental bodies, who are subject to stringent constitutional and legislative constraints, these corporations face fewer legal obligations, resulting in a greater likelihood of mishandling personal data, as exemplified by the Cambridge Analytica scandal that erupted 2018 \parencite{harbath_cambridge_2023}. Social constructivism \textemdash\ a political theory that suggests social reality shapes human behavior and one must critically assess if these actions yield beneficial outcomes for the population \textemdash\ provides a meaningful perspective on the contradiction, highlighting that public reactions to surveillance may be attributed to socially constructed norms rather than objective assessment of dangers \parencite{courseCompanion2024}. 
	
	Faced with the discrepancy of interests, both Reuter and Wandrey ultimately converged on the central idea of \textit{proportionality} as the standard by which to evaluate the balance collective security and individual rights. Fundamentally, the idea of \textit{proportionality} dictates that surveillance may only be employed when supported by sufficient justification and may not extent further than to fulfill the goal of public safety. This concept is also reflected in German legislation at a national level through the \textit{Übermaßverbot} (Prohibition of Excessive Measures) principle which states that the greater the restriction placed on personal freedom, the more substantial the public interest must be to justify it \parencite{wetzling2023}.
	
	Approaching the dilemma from different political perspectives, each with its own set of assumptions and values, underscored the multi-dimensional nature of the surveillance debate and crystallized the understanding that no definite answer can be established. As frequently observed in global politics, ideological divergences lead to conflicting interests, necessitating the pursuit of a negotiated compromise, balancing the fundamental interests of all stakeholders. In this instance, this negotiated compromise seems to be the approach of \textit{proportionality}; as suggested by both interviewees. While there is a surveillance system in place, safeguarding public security, there is a clear attempt to strengthen the civil freedoms of each individual to the greatest extent possible \textemdash\ considering the core values of both realism and liberalism. In the light of the fact that \textit{proportionality} requires justification, this idea also aligns with social constructivism. 
	
	When applying the principle of proportionality to assess the balance in Darmstadt, it becomes evident that the equilibrium between collective security and individual rights is upheld, particularly in the context of the regulations specific to Darmstadt provided in the interview with Wandrey. While public surveillance restricts certain individual liberties, it also servers to uphold others. At first, I too was primarily concerned with the implications for individual privacy and the various freedoms the populace hold within Germany. However, further involvement in the topic through the engagement activities revealed the stark necessity of surveillance for security. It seems that in an age marked by heightened concerns over privacy and emphasis on fundamental freedoms and non-discrimination, security and the right to \textit{Körperliche Unversehrheit} is widely regarded as self-evident. Therefore surveillance is frequently subject to harsh criticism on the basis of societal norms. 
	
	Yet, in order to fully assess the balance portrayed, a critical evaluation of the perspectives consulted is necessary to reflect on potential imitations in scope and neutrality. Upon reflection, it becomes evident that this essay may exhibit some degree of bias. The inclusion of Reuter provided an impartial academic perspective, whereas the interview with Wandrey offered an insight into the rationale behind policy decisions \textemdash\ both providing deeply informed viewpoints that revealed unexpected considerations. However, these are both institutional perspectives. In order to provide a more neutral evaluation of the issue, it would have been beneficial to consult members of the public, acknowledging that there might be discrepancies between institutional claims and public perception. The deliberate incorporation of a spokesperson for a minority rights organization may have provided a more concrete understanding of the possible discriminatory implications of surveillance, whereas engaging with the viewpoint of a representative of a privacy advocate group could have more thoroughly challenged the relatively favorable stance on surveillance exhibited by both interviewees. 
	
	
	
	
	\clearpage
	
	\printbibliography
	
	\clearpage
	
	\begin{appendices}
		\section{Map of Luisenplatz}
		\label{sec:luisenplatz-map}
		
		\begin{figure}[H]
			\centering
			\includegraphics[width=0.8\textwidth]{images/vü-luisenplatz-darmstadt.png} 
			\caption{Map of Luisenplatz (Darmstadt) with marked positions of the video surveillance cameras \parencite{magistratsbeschluss_luisenplatz_2020}.}
		\end{figure}
		
		\clearpage
		
		\section{Interviews}
		\label{app:interviews}
		
		\subsection{Interview Questions}
		
		\begin{figure}[H]
			\begin{enumerate}[itemsep=0.1em, topsep=0.1em]
				\footnotesize
				\item How would you, as an expert in the field of security and technology, describe the relationship between public surveillance and collective security in general?
				\item What typical areas of tension do you see between the protection of individual rights – in particular privacy – and security policy objectives in public space?
				\item Which local factors – such as technology, urban structure, or societal culture – in your view influence surveillance practices in Darmstadt?
				\item Would you say that Darmstadt currently maintains a balanced relationship between security interests and the protection of individual rights – or is this balance increasingly under pressure?
				\item Have there been situations in the recent past in which this balance was publicly discussed or called into question?
				\item In your observation, how does the presence of public surveillance influence the behavior of people in Darmstadt – for example, during demonstrations or political gatherings?
				\item Do you observe a so-called ``Chilling Effect'' – meaning that people change their behavior because they feel observed – or is surveillance here largely accepted?
				\item In your opinion, are the groups most affected by surveillance – such as young people or activists – sufficiently included in political decision-making processes?
				\item Which democratic control mechanisms do you consider particularly important to ensure an appropriate balance between security and fundamental rights?
				\item What risks do you see if surveillance infrastructures continue to expand without sufficient public oversight or debate?
				\item If you could make one concrete recommendation to the city – what would, in your opinion, be the most urgent improvement in this field of tension?
				\item From your perspective, is there an aspect of public surveillance – particularly in Darmstadt – that has so far received too little attention in public or political debate?
			\end{enumerate}
			\caption{List of guiding questions used for each interview. Both interviews were conducted in German and subsequently translated in English using \textcite{googletranslate}.}
		\end{figure}
		
		\subsection{Selected Interview Excerpts}
		
		\subsubsection{Interview with Paul Georg Wandrey}

		\vspace{0.8em}
		\quotecard{wandrey}{
			\textbf{Q:} What tensions do you see between individual privacy and public surveillance? \\[4pt]
			\textbf{A:} “...because people often argue with data protection, which is somewhat paradoxical, since on the other hand, never before has so much been shared on social media, and people upload all sorts of things to data giants.” \\[6pt]
			\emph{An excerpt of a particularly compelling answer given by Wandrey, translated from German into English using \textcite{googletranslate}.}
		}
		
		\vspace{1.5em}
		
		\subsubsection{Interview with Prof. Dr. Dr. Reuter}
		\vspace{0.8em}
		\quotecard{reuter}{
			\textbf{Q:} How do you see the risk that surveillance prevents people from freely expressing their opinions? \\[4pt]
			\textbf{A:} “Absolutely. On many, many levels I see this as critical. ... sometimes one just wants to speak frankly and get straight to the point ... and not have a partial quotation of five words taken out of context and possibly misinterpreted. That is, of course, the very big danger I see.” \\[6pt]
			\emph{A segment of a notably compelling response by Reuter, translated from German into English using \textcite{googletranslate}.}
		}
		
		
	\end{appendices}
	
\end{document}
