\documentclass[12pt]{article}

\usepackage[utf8]{inputenc}      
\usepackage[T1]{fontenc}          
\usepackage[a4paper, margin=2.5cm]{geometry}  % page size & margins
\usepackage{setspace}             % spacing 
\usepackage{graphicx}             % images
\usepackage{hyperref}             
\usepackage[backend=biber, style=apa]{biblatex} % citing
\usepackage{fontspec} % font

% Settings
\doublespacing
\addbibresource{references.bib}  

\setmainfont{Times New Roman}
\setmonofont{Lucida Console}[Scale=MatchLowercase]

% ===== Begin Document =====
\begin{document}
	
	\begin{titlepage}
		\centering
		
		\textbf{Global Politics SL \textemdash\ Engagement Activity}
		
		\vspace*{4cm}
		
		\textbf{Title:}\\
		Negotiating Security and Rights: Public Surveillance in Darmstadt
		
		\vspace{1cm}
		
		\textbf{RQ:}\\
		To what extent does the use of public surveillance in Darmstadt challenge the balance between collective security and individual rights?
		
		\vspace{4cm}
		
		Word count:\\
		XXXX
		
		\vfill
	\end{titlepage}
	
	
	In an increasingly digital and urbanized world, the use of public surveillance technologies in urban spaces has become widely ubiquitous. Justified by the promise of a collective security \textemdash\ the safeguarding of multiple people against threats ranging from domestic crime to international attacks \parencites{noauthor_security_2025}{noauthor_collective_2025} \textemdash\ these measures are particularly employed in areas marked by high foot traffic. However, the proliferation of these surveillance systems provokes concerns about the integrity of individual rights \textemdash\ in particular the right to privacy, to assembly and associate with others and the freedom of expression \parencite{nandy2023}. Striking an equilibrium between these seemingly incongruous concerns remains a central challenge in modern democracies, where freedom of expression \textemdash\ one of democracy's fundamental cornerstones \parencite{noauthor_special_nodate} \textemdash\ stands vulnerable. 
		
	Germany, embossed by its historical legacy of state surveillance \textemdash\ under the Nazi regime \parencite{mdrde_uberwachung_nodate} and the GDR \parencite{lichter_loeffler_siegloch2016} \textemdash\ has developed a multifaceted and intricate legal framework governing this issue. Within this context, Darmstadt emerges as an ideal case study.
	
	Home to the Luisenplatz \textemdash\ the central square and primary transfer hub for the area's public transportation network \textemdash\ Darmstadt embodies the urban environment where surveillance is most prominent. I have personally traversed this square countless times \textemdash\ on my way to meet up with friends or to go shopping in the vast variety of stores offered \textemdash\ and became increasingly conscious of the surveillance cameras. Realizing that any person residing on the square was being recorded without explicit consent led me to question the legitimacy of these systems \textemdash\ culminating into the research question of this paper: \textit{To what extent does the use of public surveillance in Darmstadt challenge the balance between collective security and individual rights?}.
	
	This investigation is deeply anchored in the core concepts of the Global Politics course, particularly \textit{Power} and \textit{Legitimacy}. The implementation of surveillance exemplifies power of political actors in both its hard and soft forms \parencite{courseCompanion2024}\ \textemdash\ the collection of information in order to pursue objectives such as maintaining public order, and through the potential ``chilling effect'' induced by state observation \parencite{murray2024}. One is confronted with the legitimacy of the authority of these political actors, examining whether their actions retain the democratic values instilled within Germany's constitution. Furthermore, it is also related to the thematic studies \textit{Peace and Conflict} and \textit{Rights and Justice}. Proponents of surveillance argue it is essential for ensuring peace and shielding the population from violence. Nevertheless, the utilization surveillance systems must be weighed against human rights considerations.
	
	By examining both the German legislation regarding this dilemma and exploring how surveillance is handled, implemented and perceived by the public in Darmstadt, this analysis operates both on the national and local level. Nonetheless, the local level remains the focal point, with primary information gained from two interviews: an interview with Prof. Dr. Dr. Christian Reuter, Chair of PEASEC \textemdash\ an organization which merges computer science with peace and security studies \parencite{noauthor_peasec_2025} \textemdash\ and professor at the Technische Universität Darmstadt, and Paul Georg Wandrey, Chair of the Darmstadt City Council.
	
	The interview with Reuter was chosen due to his vast expertise concerning the intersection of technology, peace and security, and democratic values specific to Darmstadt, allowing for a profound insight on the justifications and ramifications of surveillance. In contrast to Wandrey, Reuter is independent of political interests, therefore providing a critical yet neutral perspective on the matter. While highlighting the importance of proportionality, Reuter argued that a perfect balance is challenging to define due to the difficulty to measure the importance of values. However, a lack of specific details prevented him from providing concrete statements.
	
	The interview with Wandrey was selected due to his position at the forefront of the local decision-making process, providing crucial insight into the political rationale behind the use of surveillance. As part of the City Council, Wandrey has access to precise information regarding the implementation of and feedback concerning surveillance in Darmstadt \textemdash\ offsetting the shortcomings regarding concrete information of the interview with Reuter. In agreement with Reuter, Wandrey emphasized the concept of proportionality and supported regulated artificial intelligence driven surveillance systems. Moreover, he highlighted the paradox of usage of social media and explained measures facilitating an elevated level of privacy.
	
	Both interviews were video conferences conducted via Microsoft Teams \textemdash\ in exchange for in-person interviews \textemdash\ acknowledging its flexibility and greater time efficiency, improving the likelihood of securing an interview. A list of questions formulated before the interviews acted as guidance. The interviews were recorded \textemdash\ with consent \textemdash\ and transcribed for later reference.
	
	At face value, the mere employment of surveillance undermines various inherent human rights. The collection of private data, interaction logs and activity patterns through surveillance partially or even fully undermines an individual's right to privacy, challenging Articles 1(1) and 10(1) of the \textit{Grundgesetz} (German Basic Law) \parencites{nandy2023, wetzling2023, grundgesetz2025}. Furthermore, the omnipresent nature of surveillance may lead to ``self-censor[ship]'', suppressed public dialog and the restraining of free and open discourse \textemdash\ ``chilling effect'' \textemdash\ and contests Article 5(1) by compromising the right of freedom of expression \parencite{murray2024}. A reluctance to participate in non-violent demonstrations or civil society organizations may develop out of the concern of becoming a potential target \textemdash\ curtailing the right to assemble and associate with others under Articles 8(1) and 9(1). Lastly, surveillance systems may exhibit structural bias \textemdash\ possibly diminishing principles of equality and non-discrimination given by Article 3(3). When viewed through the lens of liberalism, especially the interpretation proposed by Judith Shklar \textemdash\ liberalism's main objective is to ensure a political landscape in which individuals are able to exercise their personal freedom \textemdash\ such surveillance systems raise existential concerns about the erosion of the very freedoms the state is meant to protect \parencite[684]{bell2014}.
	
	To expand on the notion of the ``chilling effect'', Reuter points out a thought-provoking implication of surveillance: fabrication of distorted realities. Partial quotes or brief excerpts of surveillance footage could be employed in order to create false narratives revolving around an individual or group of people. Beyond the use of false narratives to diminish public perception of a given target, they may serve as basis for unjust incrimination. This potential to manipulate events intensifies the ``chilling effect'', as the concern about being subject to targeted scrutiny rises significantly. 
	
	However, as Reuter also notes, it is quintessential to also consider the potentially advantageous impact of surveillance on certain individual liberties. Article 2(2) of the Grundgesetz guarantees the right to \textit{Körperliche Unversehrheit} (Physical Integrity), protecting individuals from bodily harm, torture and other infringements on physical integrity \parencite{bildung_korperliche_nodate}. Given that surveillance primarily functions to both deter and investigate criminal activity \textemdash\ as underscored by Wandrey \textemdash\ it serves to strengthen the collective security of the populace, thereby affirming the protection of the right to \textit{Körperliche Unversehrheit}. Placing heightened focus on security is in line with defensive realism \parencite{courseCompanion2024}, a political theory opposing liberalism \parencite{jumarang2011}.  
	
	A particularly compelling and often overlooked argument mentioned by Wandrey concerns the paradoxical relationship between public surveillance and the widespread usage of social media. In Germany, the majority of the population use social media platforms \parencite{koptyug2025, worldometer_germany2025}, and by doing so, voluntarily disclosing vast amounts of personal data to multinational technology giants such as Meta. Unlike governmental bodies \textemdash\ who are subject to stringent constitutional and legislative constraints \textemdash\ these corporations face fewer legal obligations, resulting in a greater likelihood of mishandling personal data, as exemplified by the Cambridge Analytica scandal that erupted 2018 \parencite{harbath_cambridge_2023}. Social constructivism provides a meaningful perspective on the contradiction, highlighting that public reactions to surveillance may be attributed to socially constructed norms rather than objective assessment of dangers \parencite{courseCompanion2024}. 
	
	Faced with the discrepancy of interests, both Reuter and Wandrey ultimately converged on the central idea of \textit{proportionality} as the standard by which to evaluate the balance collective security and individual rights. Fundamentally, the idea of \textit{proportionality} dictates that surveillance may only be employed when supported by sufficient justification and may not extent further than to fulfill the goal of public safety. This concept is also reflected in German legislation at a national level through the \textit{Übermaßverbot} (Prohibition of Excessive Measures) principle which states that the greater the restriction placed on personal freedom, the more substantial the public interest must be to justify it \parencite{wetzling2023}.
	
	When applying the principle of proportionality to assess the balance in Darmstadt, it becomes evident that the equilibrium between collective security and individual rights is upheld \textemdash\ especially when considering the regulations specific to Darmstadt provided in the interview with Wandrey. Firstly, with a daily foot traffic of well above 100,000 individuals \textemdash\ a figure provided by Wandrey \textemdash\ Luisenplatz can be considered a major public hotspot \textemdash\ providing justification given the amplified probability of criminal activities. The requirement to justify surveillance measures every two years ensures that they are reflective of the present circumstances and evolving societal needs. In an attempt to combat the ``chilling effect'', demonstrations or assemblies can be registered \textemdash\ without the need for approval \textemdash\ which automatically triggers the deactivation of surveillance through a physical cover during the registered time period. Hospitality zones are not surveilled during operation hours, providing their customers with elevated privacy. Lastly, there are defined periods for which the surveillance footage may be stored and access to it is heavily restricted, further promoting the privacy of the individual. While the surveillance system is in place, there is a clear attempt to negotiate a balance between collective security and the protection of civil liberties, adhering to the notion of \textit{proportionality}.
	
	To conclude, although at first glance public surveillance undermines certain individual rights, it is important to consider the entire picture \textemdash\ the balance \textemdash\ rather than an individual's liberties in isolation. 
	
	
	
	
	
	
	
	
	

	

	
	
	more asking
	
	
	
		
	
	
	
	
	
	
	
	
	
	\clearpage
	
	\printbibliography
	
\end{document}
